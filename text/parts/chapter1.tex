\chapter{Обзор существующих методов}
\label{chapter_review}

Формально имеется набор $\{v_1, ..., v_n\}$ из $n$ параметров ЭА, каждый из которых может принимать $\{v_{i1}, .., v_{im}\}$ значения. Это могут быть как дискретные значения, так и интервалы значений. Целью алгоритма является выбор таких значений параметров $v_i$, чтобы повысить эффективность ЭА.

Большинство методов адаптивной настройки параметров ЭА можно отнести к классу сопоставителей вероятностей (probability matching techniques), в которых вероятность выбора значения параметра пропорциональна его качеству.

\section{Метод, предложенный Karafotias}
Метод настройки параметров с помощью обучения с подкреплением.

\subsection{Модель UTree}

\subsubsection{Тест Колмогорова-Смирнова}

\subsubsection{Q-обучение}

\section{Метод Earpc}

